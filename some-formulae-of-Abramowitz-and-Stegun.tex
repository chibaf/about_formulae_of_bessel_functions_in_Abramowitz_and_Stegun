\documentclass[12pt]{report}

\input{./setting.tex}

\begin{document}
\bibliographystyle{acm}
%\bibliographystyle{acm}

\title{some Bessel asymptotic formulae of Abramowitz and Stegun}
\author{CHIBA, Fumihiro}
%------------------------
\maketitle
%\abstract{
%%
%}

%\begin{rem}
In Abramowitz and Stegun\cite{Abramowitz-Stegun}, the following formulae have no reference.
$$
J_\nu(z) \sim \sqrt{\frac{1}{2\pi\nu}}\left(\frac{ez}{2\nu}\right)^\nu\left[1+O(\nu^{-1})\right], \quad |{\rm arg}\, \nu|<\pi
$$
$$
Y_{\nu}(z) \sim -\sqrt{\frac{2}{\pi\nu}}\left(\frac{ez}{2\nu}\right)^{-\nu}\left[1+O(\nu^{-1})\right], \quad |{\rm arg}\, \nu|<\pi.
$$

\label{Debye formula}
Using Debye's contour, Debye shows the following asymptotic expansions (\ref{debye-formula: Hankel}) and (\ref{debye-formula: Bessel}) of order $\alpha$ for $H^{(2)}_{\alpha}(x)$ and $J_\alpha(x)$ from their integral representations\cite{Debye},\cite{Olver1997},\cite{Watson1966}, where $\alpha>x>0$ and $H^{(2)}_{\alpha}(x)$ is the $\alpha$th order Hankel function of the second kind. Watson gives an explanation on Debye's contour in his book\cite{Watson1966}.
\begin{eqnarray}
H^{(2)}_{\alpha}(x)&\sim&\frac{\mathrm{i}}{\pi}\e^{-\mathrm{i}x(\sin\tau_0-\tau_0\cos\tau_0)}\left[\frac{\Gamma(\frac{1}{2})}{(\mathrm{i}\frac{x}{2}\sin\tau_0)^{\frac{1}{2}}}-\left(\frac{1}{8}+\frac{5}{24}\cot^2\tau_0\right)\frac{\Gamma(\frac{3}{2})}{(\mathrm{i}\frac{x}{2}\sin\tau_0)^{\frac{3}{2}}}\right. \nonumber\\
&&\left.+\left(\frac{3}{128}+\frac{7}{576}\cot^2\tau_0+\frac{385}{3456}\cot^4\tau_0\right)\frac{\Gamma(\frac{5}{2})}{(\mathrm{i}\frac{x}{2}\sin\tau_0)^{\frac{5}{2}}}+\cdots\right],
\label{debye-formula: Hankel}
\end{eqnarray}
where $\tau_0$ is a saddle point and defined through
\begin{equation}
\tau_0=-\mathrm{i}\log\left(\frac{\alpha}{x}+\frac{\alpha}{x}\sqrt{1-\left(\frac{x}{\alpha}\right)^2}\right).
\label{tau_0}
\end{equation}
Since 
\begin{displaymath}
\cos\tau_0=\frac{\alpha}{x}, \quad \sin\tau_0=-\mathrm{i}\frac{\alpha}{x}\sqrt{1-\left(\frac{x}{\alpha}\right)^2},
\end{displaymath}
we have for a fixed $x>0$
\begin{eqnarray*}
\e^{-\mathrm{i}x(\sin\tau_0-\tau_0\cos\tau_0)}&=&\exp\left\{-\alpha\sqrt{1-\left(\frac{x}{\alpha}\right)^2}+\alpha\log\left(\frac{\alpha}{x}+\frac{\alpha}{x}\sqrt{1-\left(\frac{x}{\alpha}\right)^2}\right)\right\} \\
&=&\exp\left(-\alpha\sqrt{1-\left(\frac{x}{\alpha}\right)^2}\right) \times \left(\frac{\alpha}{x}+\frac{\alpha}{x}\sqrt{1-\left(\frac{x}{\alpha}\right)^2}\right)^\alpha \\
&\sim& \e^{-\alpha} \times \left(\frac{2\alpha}{x}\right)^\alpha=\left(\frac{2\alpha}{\e x}\right)^\alpha \quad {\rm as}\;\alpha\rightarrow\infty.
\end{eqnarray*}
On the other hand, the following formulae hold for a fixed $x>0$.
\begin{displaymath}
\Gamma\left(\frac{1}{2}\right)=\sqrt{\pi} \quad {\rm and} \quad \left(\mathrm{i}\frac{x}{2}\sin\tau_0\right)^{1/2} \sim \sqrt{\frac{\alpha}{2}}\quad {\rm as}\;\alpha\rightarrow\infty.
\end{displaymath}
Then the first term of (\ref{debye-formula: Hankel}) is asymptotically equal to
\begin{displaymath}
\mathrm{i}\sqrt{\frac{2}{\pi\alpha}}\left(\frac{2\alpha}{\e x}\right)^\alpha.
\end{displaymath}
Hence we have
\begin{displaymath}
H^{(2)}_\alpha(x) \sim \mathrm{i}\sqrt{\frac{2}{\pi\alpha}}\left(\frac{2\alpha}{\e x}\right)^\alpha \quad {\rm as}\;\alpha\rightarrow\infty.
\end{displaymath}

In the same manner as this discussion, from the following Debye's formula\cite{Debye} we have the asymptotic expansion of $J_\alpha(x)$ for a fixed positive number $x$ as $\alpha\rightarrow\infty$.
\begin{eqnarray}
J_{\alpha}(x)&\sim&\frac{\mathrm{1}}{\pi}\e^{\mathrm{i}x(\sin\tau_0-\tau_0\cos\tau_0)}\left[\frac{\Gamma(\frac{1}{2})}{(\mathrm{i}\frac{x}{2}\sin\tau_0)^{\frac{1}{2}}}+\left(\frac{1}{8}+\frac{5}{24}\cot^2\tau_0\right)\frac{\Gamma(\frac{3}{2})}{(\mathrm{i}\frac{x}{2}\sin\tau_0)^{\frac{3}{2}}}\right. \nonumber\\
&&\left.+\left(\frac{3}{128}+\frac{7}{576}\cot^2\tau_0+\frac{385}{3456}\cot^4\tau_0\right)\frac{\Gamma(\frac{5}{2})}{(\mathrm{i}\frac{x}{2}\sin\tau_0)^{\frac{5}{2}}}+\cdots\right],
\label{debye-formula: Bessel}
\end{eqnarray}
where $\tau_0$ is a saddle point and defined through
\begin{equation}
\tau_0=-\mathrm{i}\log\left(\frac{\alpha}{x}-\frac{\alpha}{x}\sqrt{1-\left(\frac{x}{\alpha}\right)^2}\right).
\label{tau_0-2}
\end{equation}
%\end{rem}

\cleardoublepage
\addcontentsline{toc}{chapter}{Bibliography}
\bibliography{biblio-2008}


\end{document}

